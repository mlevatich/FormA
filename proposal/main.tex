\documentclass{article}

\usepackage{xspace}
\usepackage{xcolor}
\usepackage{amsmath}
\usepackage{amssymb}
\usepackage{amsthm}
\usepackage{amsmath}
\usepackage{graphicx}

\newtheorem{conjecture}{Conjecture}[section]
\newtheorem{corollary}{Corollary}
\newtheorem{definition}{Definition}[section]
\newtheorem{example}{Example}

\newcommand{\LTLglobally}{\Box}
\newcommand{\LTLeventually}{\Diamond}
\newcommand{\LTLnext}{\bigcirc}
\newcommand{\Name}{\textit{GameTitle}}

\title{COMS 6863 Final Project Proposal: \Name{}}
\author{Max Levatich \& Wonhyuk (Harry) Choi}
\date{\today}

\begin{document}
\maketitle

\section{Introduction}

    Our aim is to design, program, and verify via model-checking a lightweight
    video game called \Name{}. \Name{} is based loosely on (Asteroids? Pong?
    Mario?); the player (briefly describe how the game is played).

    \Name{} will be implemented in C, with the help of the low-level rendering
    library SDL. We will use CBMC \cite{clarke2004tool} to model-check
    certain properties of the game as we develop it, including collision
    detection, collision resolution, and input correspondence. We will submit:

    \begin{enumerate}
        \item{The game's code and assets}
        \item{Build instructions, including how to invoke CBMC}
        \item{A correctness specification of the properties checked}
        \item{A final project report detailing our process}
    \end{enumerate}

\section{Game Overview}

    \subsection{Implementation}

        \Name{} is written in C (C11 standard) and depends on Simple DirectMedia
        Layer 2.0 (SDL). SDL is a development library written in C which
        provides a simple low-level interface to the keyboard, audio, and
        graphics hardware necessary to program a video game. We will use CBMC
        to model-check the game (see Section 4).

        We will use the traditional game programming technique of a \textit{game
        loop} located in the main file, where each iteration of the loop
        represents one \textit{frame}, in which we read user input, update the
        game state, and render the game state to the screen. Modules and
        functions will be designed specifically to provide a minimal interface
        to the model-checked portions of our code. We aim to have verified
        modules which encapsulate specific features of the game (e.g. collision
        detection, keyboard input).

    \subsection{Mechanics}

        TODO: Describe how the game is played in detail, with emphasis on the
        core algorithms like collision detection and resolution that we model
        check.

\section{Correctness Specification}

    \subsection{Collision Handling}

        TODO: LTL formulas and descriptions here!

    \subsection{Input Correspondance}

        TODO: LTL formulas and descriptions here!

\section{Verification Approach}

    TODO: Describe CBMC, describe how we will use it as we develop the game.

\bibliographystyle{IEEEtran}
\bibliography{proposal}

\end{document}
